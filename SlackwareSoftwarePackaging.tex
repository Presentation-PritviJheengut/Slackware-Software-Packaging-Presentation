\documentclass[12pt,hyperref={pdfpagelabels=true}]{beamer}

%% License to be introduced

\usepackage[T1]{fontenc}
\usepackage[utf8]{inputenc}
\usepackage{hyperref}
\usepackage{pgfarrows,pgfnodes,pgfautomata,pgfheaps,pgfshade}
\usepackage{listings}
\usepackage[all]{xy}

\sffamily

%\pdfinfo
%{
%  /Title       (Slackware Software Packaging: A How To)
%  /Creator     (Pritvi Jheengut)
%  /Author      (Pritvi Jheengut)
%}

\title{Slackware Software Packaging: How and Where To Start}
\subtitle{How many of you are aware of the steps in creating Slackware packages}
\author{Pritvi Jheengut}
\date{July 2016}

\usetheme{Rodrigues}

\mode<presentation>{}

\begin{document}

\frame{\titlepage}

\begin{frame}
This work is licensed under the Creative Commons Attribution 4.0 International 
License. To view a copy of this license, visit \href{
http://creativecommons.org/licenses/by/4.0/} or
send a letter to
Creative Commons, 
PO Box 1866,
Mountain View, 
CA 94042,
USA.
\end{frame}

\section*{tableofcontent}

\begin{frame}
  \frametitle{Outline}
  \framesubtitle{Table of Contents}
  \tableofcontents[hidesubsections]
\end{frame}

\AtBeginSection[]{
  \frame<handout:0>{
    \frametitle{Outline}
    \tableofcontents[currentsection,hideallsubsections]
  }
}

\AtBeginSubsection[]{
  \frame<handout:0>{
    \frametitle{Outline}
    \tableofcontents[sectionstyle=show/hide,subsectionstyle=show/shaded/hide]
  }
}

\newcommand<>{\highlighton}[1]{%
  \alt#2{\structure{#1}}{{#1}}
}

\newcommand{\icon}[1]{\pgfimage[height=1em]{#1}}

%%%%%%%%%%%%%%%%%%%%%%%%%%%%%%%%%%%%%%%%%
%%%%%%%%%% Content starts here %%%%%%%%%%
%%%%%%%%%%%%%%%%%%%%%%%%%%%%%%%%%%%%%%%%%

\section{Introduction}

\begin{frame}
  \frametitle{Prerequisites}
  \framesubtitle{The Shell and Coreutils}
  \begin{block}{Shell}
    \begin{itemize}[<+-| alert@+>]
    \item Some basic shell programming skills
    \item GNU Coreutils
    \end{itemize}
  \end{block}

  \pause

  \begin{example}{GNU Coreutils}
    \begin{itemize}[<+-| alert@+>]
    \item mkdir
    \item cat
    \item ls
    \end{itemize}
  \end{example}
\end{frame}

\begin{frame}
  \frametitle{Prerequisites}
  \framesubtitle{Packaging tools}
  \begin{block}{Unix}
    \begin{itemize}[<+-| alert@+>]
    \item tar
    \item cp/mv/install
    \item wget/curl
    \item chmod/chown
    \item mktemp/mkdir
    \item strip
    \item autotools/make/cmake/python/perl/ruby/
    \item find/xargs/grep
    \end{itemize}
  \end{block}
\end{frame}

\begin{frame}
  \frametitle{Prerequisites}
  \framesubtitle{Compression Utilities}
  \begin{columns}
    \begin{column}{4cm}
      \begin{block}{Compression Utility}
        \begin{description}
        \item<1-> gzip
        \item<2-> bzip2
        \item<3-> xz/lzma
        \end{description}
      \end{block}
    \end{column}

    \begin{column}{4cm}
      \begin{block}{mime}
        \begin{itemize}
        \item<1-> tgz
        \item<2-> tbz
        \item<3-> txz
        \end{itemize}
      \end{block}
    \end{column}
  \end{columns}
\end{frame}

\begin{frame}
  \frametitle{Prerequisites}
  \framesubtitle{Slackware Basic tools}
  \begin{block}{Slackware}
    \begin{itemize}[<+-| alert@+>]
    \item<1-7> pkgtool - The Package Manager
    \item<2>  \url{http://tukaani.org/pkgtools/} an old forked version
    \item<3-7> installpkg
    \item<4-7> removepkg
    \item<5-7> upgradepkg
    \item<6-7> explodepkg
    \item<7-7> makepkg - The Package Creator/Maker
    \item<8-> slackbuild scripting
    \end{itemize}
  \end{block}
\end{frame}

\section{What is a Package?}

\begin{frame}
  \frametitle{Definition of a package}
  \framesubtitle{A Slacker's goal to world domination}
  \begin{block}{A Package}
    \begin{description}
    \item[Definition:] An archive format with/out instructions used for the
      installation of files on a system
    \end{description}
  \end{block}
  
  \pause
  
  \begin{columns}
    \begin{column}{4cm}
      \begin{block}{Example}
        \begin{itemize}[<+-| alert@+>]
        \item<2-> t?z
        \item<3-> deb
        \item<4-> rpm
        \end{itemize}
      \end{block}
    \end{column}
    
    \begin{column}{4cm}
      \begin{block}{mime}
        \begin{itemize}[<+-| alert@+>]
        \item<2-> Slackware
        \item<3-> Debian
        \item<4-> Red Hat
        \item<4-> OpenSUSE
        \end{itemize}
      \end{block}
    \end{column}
  \end{columns}
\end{frame}

\begin{frame}
  \frametitle{Package Manager}
  \framesubtitle{How to install Packages}
  \begin{block}{Package Manager}
    A package manager is a collection of software tools that
    \begin{itemize}[<+-| alert@+>]
    \item<2,7>install
    \item<3,7> upgrade
    \item<4,7> configure
    \item<5,7> remove
    \item<6,7> performs other related jobs
    \end{itemize}
    
    \pause
    
    on software packages for a Linux Distribution.
  \end{block}
\end{frame}

\section{Preparation For Building}

\begin{frame}[fragile]
  \frametitle{Prepare The Build Directory}
  \framesubtitle{Sanitize The Build Directory}
  \lstset{language=sh}
  \begin{lstlisting}
    mktmp
    cd /tmp/$"The Temporary File Just Created"
  \end{lstlisting}
  
  \pause
  
  \begin{alertblock}{\href{http://unixhelp.ed.ac.uk/CGI/man-cgi?mktemp}{mktmp}}
    Create temporary files/directories that are predictable, safe and to avoid
    possible race conditions.
  \end{alertblock}
  
  \pause

  \begin{alertblock}{/tmp}
    The /tmp directory is assumed to cleaned regularly, sometimes assumed a
    tmpfs is mount over tmp
    
    The source code is often extracted into /tmp directly.
  \end{alertblock}
\end{frame}

\begin{frame}[fragile]
  \frametitle{Prepare The Build Directory}
  \framesubtitle{Putting The Source Code in /tmp}
  \lstset{language=sh}
  \begin{block}{Copy The Source Code}
    Copy the source code archive into the tmp dir
  \end{block}
  
  \pause
  
  \begin{lstlisting}
    cp ${old path}/${source code archive} .
  \end{lstlisting}
  
  \pause
  
  \begin{block}{Download The Source Code}
    Or Download the source code archive into the tmp dir
  \end{block}
  
  \pause
  
  \begin{lstlisting}
    wget/curl
    protocol://url/${source code archive}
  \end{lstlisting}
\end{frame}

\begin{frame}[fragile]
  \frametitle{Populating The Build Directory}
  \framesubtitle{Extracting the source code into The Build Directory}
  \lstset{language=sh}
  
  \pause
  
  \begin{alertblock}{Extracting The Source Code}
    This step is optional
  \end{alertblock}
  
  \pause
  
  \begin{lstlisting}
    mkdir build
    cd build
  \end{lstlisting}
  
  \pause
  
  \begin{block}{Extracting The Source Code}
    Extract files into the `mktmp`/tmp directory
  \end{block}
  
  \pause
  
  \begin{lstlisting}
    tar -xvvf ../"${source code archive}"
  \end{lstlisting}

  \pause

  \begin{equation*}
    \xymatrix@C=2pc@R=2pc{
      \//tmp\ar@{->>}[r]_D & \/\$PKG\ar@{->>}[r]_D & \/build}
  \end{equation*}
\end{frame}

\section{Building The Source Code}

\begin{frame}
  \frametitle{Configuration of the source code}
  \framesubtitle{Pre-Configuration and Configuration}
  \begin{columns}
    \begin{column}{5cm}
      \begin{block}{pre-configuration}
        Rarely is pre-configuration performed.
      \end{block}
    \end{column}

    \begin{column}{5cm}
      \begin{block}{configuration}
        Most of the times configuration is needed.
      \end{block}
    \end{column}
  \end{columns}
  
  \pause
  
  \begin{columns}
    \begin{column}{5cm}
      \begin{example}{autotools}
        \begin{itemize}[<+-| alert@+>]
        \item aclocal
        \item autoconf
        \item automake
        \end{itemize}
      \end{example}
    \end{column}
    
    \begin{column}{5cm}
      \begin{example}
        \begin{itemize}[<+-| alert@+>]
        \item ./configure
        \item cmake
        \item python setup.py
        \item gem
        \item qmake
        \end{itemize}
      \end{example}
    \end{column}
  \end{columns}
\end{frame}

\begin{frame}
  \frametitle{Building the Source Code}
  %\framesubtitle{}
  \begin{block}{Building the Source Code}
    After configuration, the code has to be compiled into machine code.
  \end{block}
  
  \pause
  
  \begin{example}
    \begin{itemize}[<+-| alert@+>]
    \item make
    \item perl Makefile.pl
    \item gem
    \item python setup.py build
    \end{itemize}
  \end{example}
\end{frame}

\begin{frame}
  \frametitle{Building the Source Code}
  \framesubtitle{Summary}
  \begin{block}{A Summary}
    A summary of the job done
  \end{block}
  \begin{equation*}
    \xymatrix@C=2pc@R=2pc{
      Pre-Configuration\ar@{->>}[r] & Configuration\ar@{->>}[r] & Building}
  \end{equation*}
\end{frame}

\section{Creating The Package}

\begin{frame}[fragile]
  \frametitle{The Creation of a Package}
  \framesubtitle{Populating the package directory}
  \begin{block}{Installation of Package Files in tmp directory}
    The package files are installed in a directory using the
    $PKG=\$TMP/package-\$PRGNAM$ shell path parameter.
  \end{block}
  
  \pause
  
  \begin{lstlisting}
    make install DESTDIR=$PKG
    python setup.py install --root=$PKG
    gem install --install-dir $PKG/$DESTDIR
  \end{lstlisting}
  
  \pause
  
  \begin{equation*}
    \xymatrix@C=2pc@R=2pc{
      Pre-Configuration\ar@{->>}[r] & Configuration\ar@{->>}[r] &
      Building\ar@{->>}[r] & Install}
  \end{equation*}
\end{frame}

\begin{frame}[fragile]
  \frametitle{The Creation of a Package}
  \framesubtitle{Performing the usual Slackware Business}
  \begin{block}{stripping of binaries}
    After installation of object files/binaries, they need to be stripped off of
    unnecessary symbols.
  \end{block}
  
  \pause
  
  \begin{lstlisting}
    find $PKG -print0 | xargs -0 file \
    | grep -e "executable" -e "shared object" \
    | grep ELF | cut -f 1 -d : \
    | xargs strip --strip-unneeded \
    2> /dev/null || true
    
  \end{lstlisting}
  
\end{frame}

\begin{frame}
  \frametitle{The Creation of a Package}
  \framesubtitle{Performing the usual Slackware Business}
  
  \begin{block}{strip and the rest}
    stripping is one of a few more steps in making a more acceptable package,
    others include compressing the man and info pages, removing excess data and
    adding documentation
  \end{block}
  
  \pause
  
  \begin{equation*}
    \xymatrix@C=2pc@R=2pc{
      Pre-Conf\ar@{->}[r] & Conf\ar@{->>}[r] &
      Build\ar@{->>}[r] & Install\ar@{->>}[d]\\
      & Doc & CleanUP\ar@{->>}[l] & Strip\ar@{->>}[l]}
  \end{equation*}
  
\end{frame}

\begin{frame}[fragile]
  \frametitle{The Creation of a Package}
  \framesubtitle{install directory}
  \begin{block}{\$PKG/install directory}
    An extra install directory exist in all Slackware Packages.
    This folder normally contains two files.
    \begin{itemize}[<+-| alert@+>]
    \item slack-desc : This contains a summary and full description of the
      package.
    \item doinst.sh : contains instructions for the post installation stage,
      most of the time linking of files.
    \end{itemize}
  \end{block}

  \pause
  
  \begin{equation*}
    \xymatrix@C=2pc@R=2pc{
      \$PKG\/\ar[r]_D\ar[d]_D & install\ar[r]_F\ar[d]_F & slack-desc\\
      OtherDir  & doinst.sh  }
  \end{equation*}
\end{frame}

\begin{frame}[fragile]
  \frametitle{The Creation of a Package}
  \framesubtitle{install directory}
  \begin{block}{\$PKG/install directory}
    Copy the slack-desc and a custom doinst.sh if necessary into ./install
  \end{block}
  
  \pause
  
  \begin{lstlisting}
    mkdir -p $PKG/install
    
    cat $CWD/slack-desc > \
    $PKG/install/slack-desc
    
    cat $CWD/doinst.sh > \
    $PKG/install/doinst.sh
  \end{lstlisting}
\end{frame}


\begin{frame}[fragile]
  \frametitle{The Creation of a Package}
  \framesubtitle{Using makepkg}
  \begin{block}{Makepkg}
    makepkg creates a Slackware compatible package. The package is constructed
    using the contents of the current directory and all sub directories.
  \end{block}
  
  \pause
  
  \begin{lstlisting}
    cd $PKG
    makepkg $PRGNAM.tgz
  \end{lstlisting}
\end{frame}

\begin{frame}
  \frametitle{The Creation of a Package}
  \framesubtitle{makepkg options}
  \begin{example}{makepkg options}
    \begin{itemize}[<+-| alert@+>]
    \item option -l :: add any symbolic links found to the install script
      (doinst.sh) and delete them.
    \item option -c :: makepkg will reset all directory permissions to 755 and
      ownership to root:root
    \end{itemize}
  \end{example}
\end{frame}

\begin{frame}[fragile]
  \frametitle{The Creation of a Package}
  \framesubtitle{makepkg options}
  \begin{block}{Makepkg}
    with makepkg full options
  \end{block}
  
  \begin{lstlisting}
    /sbin/makepkg \
    -l y \
    -c n \
    /tmp/$PRGNAM-$VERSION-$ARCH-$BUILD$TAG.tgz
  \end{lstlisting}
  
  \pause

  \begin{block}{Package}
    This is it, your package is ready and waiting for you in /tmp.
  \end{block}
\end{frame}


\begin{frame}[fragile]
  \frametitle{Slackbuilds}
  \framesubtitle{Scripting}
  \begin{block}{Slackbuilds}
    Want to Automate your package building, then the only sane way is
    maintaining a script that we commonly call slackbuild script.
    All the steps above are entered into a script and tada, its done.
  \end{block}

  \pause

  \begin{lstlisting}
    #!/bin/bash
    PKG=/tmp/package-$PRGNAM
    rm -rf $PKG
    mkdir -p $TMP $PKG
    cd /tmp
    rm -rf $PRGNAM-$VERSION
  \end{lstlisting}
\end{frame}

\begin{frame}[fragile]
  \frametitle{Slackbuilds}
  \framesubtitle{Scripting - continued}

  \begin{lstlisting}
    tar xvf $CWD/$PRGNAM-$VERSION.tar.gz
    cd $PRGNAM-$VERSION
    chown -R root:root .
    ./configure
    make
    make install DESTDIR=$PKG
    find ... | xargs strip ...
    mkdir -p $PKG/install
    cat $CWD/slack-desc > $PKG/install/slack-desc
    cat $CWD/doinst.sh > $PKG/install/doinst.sh
    cd $PKG
    /sbin/makepkg /tmp/$PRGNAM.tgz
  \end{lstlisting}
\end{frame}

\begin{frame}
  \frametitle{Slackbuilds}
  \framesubtitle{slackbuilds.org}
    \begin{block}{slackbuilds.org}
      Templates can be found at
      http://slackbuilds.org/templates
    \end{block}
\end{frame}

\section{The End}

\begin{frame}
  \frametitle{Goal}
  \framesubtitle{A Slacker's goal to world domination}
  \begin{block}{Goal}
    \begin{itemize}[<+-| alert@+>]
    \item slack the hard way
    \item sip some beer
    \item Take over the world
    \item Relax...
    \end{itemize}
  \end{block}
\end{frame}

\begin{frame}
  \frametitle{Resources}
  \framesubtitle{If you want to improve this style}
  \begin{thebibliography}{10}
    \beamertemplatearticlebibitems

  \bibitem{beamer-homepage}
    LaTeX Beamer
    \newblock {\tt http://latex-beamer.sourceforge.net/}

  \bibitem{kdeslides}
    KDE Presentations
    \newblock {\tt http://www.kde.org/kdeslides/}

  \bibitem{My Prezz}
    My Prezz at Github
    \newblock {\tt https://github.com/Bluetailedgecko/\\
    Slackware-Software-Packaging-Presentation}

  \end{thebibliography}
\end{frame}

\frame{
  \vspace{2cm}{
    \huge Questions ?
  }

  \vspace{3cm}
  \begin{flushright}
    Pritvi Jheengut

    \structure{\footnotesize{z.coldplayer@gmail.com}}
  \end{flushright}
}

\end{document}
