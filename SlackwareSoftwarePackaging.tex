\documentclass[12pt,hyperref={pdfpagelabels=true}]{beamer}

\usetheme{Air}
\usepackage[utf8]{inputenc}
\usepackage{pgfarrows,pgfnodes,pgfautomata,pgfheaps,pgfshade}
\usepackage{listings}

%\pdfinfo
%{
%  /Title       (Slackware Software Packaging: A How To)
%  /Creator     (Pritvi Jheengut)
%  /Author      (Pritvi Jheengut)
%}

\title{Slackware Software Packaging: How and Where To Start}
\subtitle{How many of you are aware of the steps in creating Slackware packages}
\author{Pritvi Jheengut}
\date{13 March 2015}

\mode<presentation>{}

\begin{document}

\frame{\titlepage}

\section*{tableofcontent}

\begin{frame}
  \frametitle{Outline}
  \tableofcontents[hidesubsections]
\end{frame}

\AtBeginSection[]{
  \frame<handout:0>{
    \frametitle{Outline}
    \tableofcontents[currentsection,hideallsubsections]
  }
}

\AtBeginSubsection[]{
  \frame<handout:0>{
    \frametitle{Outline}
    \tableofcontents[sectionstyle=show/hide,subsectionstyle=show/shaded/hide]
  }
}

\newcommand<>{\highlighton}[1]{%
  \alt#2{\structure{#1}}{{#1}}
}

\newcommand{\icon}[1]{\pgfimage[height=1em]{#1}}

%%%%%%%%%%%%%%%%%%%%%%%%%%%%%%%%%%%%%%%%%
%%%%%%%%%% Content starts here %%%%%%%%%%
%%%%%%%%%%%%%%%%%%%%%%%%%%%%%%%%%%%%%%%%%

\section{Introduction}

\begin{frame}
  \frametitle{Prerequisites}
  \framesubtitle{The Shell and Coreutils}
  \begin{block}{Shell}
    \begin{itemize}
    \item Some basic shell programming skills 
    \item GNU Coreutils
    \end{itemize}
  \end{block}
  
  \begin{example}{GNU Coreutils}
    \begin{itemize}
    \item mkdir
    \item cat
    \item ls
    \end{itemize}
  \end{example}
\end{frame}

\begin{frame}
  \frametitle{Prerequisites}
  \framesubtitle{Packaging tools}
  \begin{block}{Unix}
    \begin{itemize}
    \item<1-> tar
    \item<2-> install
    \item<3-> cp/mv
    \item<4-> chmod/chown
    \end{itemize}
  \end{block}
\end{frame}

\begin{frame}
  \frametitle{Prerequisites}
  \framesubtitle{Compression Utilities}
  \begin{description}
  \item[mime]   Compression Utility
  \item<2->[tgz]    gzip
  \item<3->[tbz]    bzip2
  \item<4->[txz]    xz/lzma
  \end{description}
\end{frame}

\begin{frame}
  \frametitle{Prerequisites}
  \framesubtitle{Slackware Basic tools}
  \begin{block}{Slackware}
    \begin{itemize}
    \item<1-> pkgtool - The Package Manager
    \item<2>  \url{http://tukaani.org/pkgtools/} an old forked version
    \item<3-> installpkg
    \item<4-> removepkg
    \item<5-> upgradepkg
    \item<6-> explodepkg
    \item<7-> makepkg - The Package Creator/Maker
    \item<8-> slackbuild scripting
    \end{itemize}
  \end{block}
\end{frame}

\section{What is a Package?}

\begin{frame}
  \frametitle{Definition of a package}
  \framesubtitle{A Slacker's goal to world domination}
  \begin{block}{A Package}
    \begin{description}
    \item[Definition:] An archive format with/out instructions used for the
      installation of files on a system
    \end{description}
  \end{block}
  
  \begin{block}{Example}
    \begin{itemize}
    \item t?z
    \item deb
    \item rpm
    \end{itemize}
  \end{block}
\end{frame}

\section{Preparation For Building}

\begin{frame}[fragile]
  \frametitle{Prepare The Build Directory}
  \framesubtitle{Sanitize The Build Directory}
  \lstset{language=sh}
  \begin{lstlisting}
    mktmp
    cd /tmp/$"The Temporary File Just Created"
  \end{lstlisting}
  
  \begin{alertblock}{\href{http://unixhelp.ed.ac.uk/CGI/man-cgi?mktemp}{mktmp}}
    Create temporary files/directoies that are predictable, safe and to avoid
    possible race conditions.
  \end{alertblock}
  
  \begin{alertblock}{/tmp}
    The /tmp directory is assumed to cleaned regularly, sometimes assumed a
    tmpfs is mount over tmp
  \end{alertblock}
\end{frame}

\begin{frame}[fragile]
  \frametitle{Prepare The Build Directory}
  \framesubtitle{Populating The Build Directory}
  \lstset{language=sh}
  \begin{alertblock}{Dump The Source Code}
    copy the source code archive into the tmp dir 
    afterwards extract files into the `mktmp` directory
  \end{alertblock}
  
  \begin{lstlisting}
    cp ${old path}/${source code archive} .
    OR
    wget/curl protocol://url/
    ${source code archive}
    mkdir build
    cd build
    tar -xvvf ../"${source code archive}"
    unzip ../"${source code archive}"
  \end{lstlisting}
\end{frame}

\begin{frame}
  \begin{example}
    \begin{itemize}
    \item 
    \item 
    \end{itemize}
  \end{example}
\end{frame}

\section{The End}

\begin{frame}
  \frametitle{Goal}
  \framesubtitle{A Slacker's goal to world domination}
  \begin{block}{Goal}
    \begin{itemize}
    \item slack the hard way
    \item sip some beer
    \item Take over the world
    \item Relax...
    \end{itemize}
  \end{block}
\end{frame}
 
\begin{frame}
  \frametitle{Resources}
  \framesubtitle{If you want to improve this style}
  \begin{thebibliography}{10}
    \beamertemplatearticlebibitems
  
  \bibitem{beamer-homepage}
    LaTeX Beamer
    \newblock {\tt http://latex-beamer.sourceforge.net/}
    
  \bibitem{kdeslides}
    KDE Presentations
    \newblock {\tt http://www.kde.org/kdeslides/}
    
  \end{thebibliography}
\end{frame}

\frame{
  \vspace{2cm}
         {\huge Questions ?}
         
         \vspace{3cm}
         \begin{flushright}
           Pritvi Jheengut
           
           \structure{\footnotesize{z.coldplayer@gmail.com}}
         \end{flushright}
}

\end{document}
